\documentclass[12pt,titlepage]{article}
\usepackage[margin=1in]{geometry}

\begin{document}
  \begin{titlepage}
    \vspace*{\fill}
    \centering

    \textbf{\Huge SE 463 Course Notes} \\ [0.4em]
    \textbf{\Large Software Requirements Specification and Analysis} \\ [1em]
    \textbf{\Large Michael Socha} \\ [1em]
    \textbf{\large 4A Software Engineering} \\
    \textbf{\large University of Waterloo} \\
    \textbf{\large Spring 2018} \\
    \vspace*{\fill}
  \end{titlepage}

  \newpage 

  \tableofcontents

  \newpage

  \section{Course Overview}
    \subsection{Logistics}
      \begin{itemize}
        \item \textbf{Professor:} Joanne Atlee
        \item \textbf{Email:} jmatlee@uwaterloo.ca
      \end{itemize}

    \subsection{Topics Covered}
      Many software project failures are not due to technical reasons, but rather due to poor project specifications. This course
      focuses on requirement-related topics including:
      \begin{itemize}
        \item Different types of requirements
        \item Learning about stakeholder needs
        \item Analyzing and refining preliminary requirements
        \item Expressing requirements in various representations
        \item Managing requirement changes
      \end{itemize}

  \section{Business Model}
    \subsection{Traditional Product Model}
      A traditional ``waterfall'' style of development relies heavily on the initial implementation meeting customer needs. This
      approach often leads to requirement-related problems, since the business and development team might not understand the
      customer needs or the customer needs might change, and a waterfall model offers little room to refine these requirements.

    \subsection{Lean Product Development Model}
      A lean product development model focuses on first determining whether a project addresses an important problem (e.g. problem
      fit), and whether the project provides a solution people want (market fit). Only after these are established is a product built,
      which is usually nimble at first and then scales up.

    \subsection{Lean Canvas}
      A lean canvas is a template to help determine the initial requirements for a project, especially one in the new product or
      consulting space. The sections of a lean canvas are described below:

      \subsubsection{Customer Segments}
        This is a list of potential customers and users. It helps to have granular segments, and focus should be placed on segments likely
        to become early adopters.

      \subsubsection{Problem Segments}
        This is a list of a few top problems that each customer segment wants solved.

      \subsubsection{Unique Value Proposition}
        This is a reason why the project is unique and valuable. The focus should also be on early adopters.

      \subsubsection{Solution}
        This is a list of solutions early adopters already apply to solve this problem. From this list, some key capabilities and features
        of the software project can be determined.

      \subsubsection{Channels}
        This is a list of methods to reach the project's target customers. This is important to consider early, since these customers
        are not just useful as end users, but also as subjects of testing and experiments during development.

      \subsubsection{Revenue Streams and Costs}
        Revenue and costs should be considered early in a project. If the project is going to operate on a subscription model, it is
        a good idea to charge something from the start (with the exception of a free trial period). 'Freemium' business models, in
        which core services are free, tend to have low conversion rates to paying features.

      \subsubsection{Key Metrics}
        This is a list of metrics to be used for measuring success.

      \subsubsection{Unfair Advantage}
        This refers to some sort of competitive advantage that is hard to replicate. Examples include an existing reputation or customer
        base, proprietary data, or patented work.

  \section{Hypothesis Testing}
    Purpose of hypothesis testing is to validate or disprove guesses in a project's business model. Hypothesis testing is generally
    conducted by performing pass/fail experiments through market research or customer interviews. Hypothesis being disproved can lead
    to learning, after which the hypothesis can be pivoted based on new understanding.

    \subsection{Hypothesis}
      A hypothesis is a tentative guess about a phenomenon of interest that is testable and falsifiable. The phenomenon of interest
      (e.g. characteristic of customer problem) are considered to be dependent variables, while independent variables affect the
      phenomenon of interest. A good hypothesis must be testable (i.e. possible to observe effects of independent variables on
      dependent variables) and falsifiable.

      \subsection{Falsifiable Hypothesis}
        A falsifiable hypothesis is a statement that can easily be proven wrong. Specific and testable statements are more falsifiable
        than vague ones. Falsifiable hypothesis often have somewhat of a universally quantification aspect (e.g. all, most, etc), rather
        than merely existential quantification.

    \subsection{Testing Risks}
      Uncertainty simply means that there are multiple possibilities. Risk tends to involve uncertainty, with some of the possibilities
      leading to failure. The three largest risks to a project are:

        \subsubsection{Customer Risk}
          e.g. Is there a viable customer segment, and what are their risks?

        \subsubsection{Product Risk}
          e.g. How do customers rank the top three problems?

        \subsubsection{Market Risk}
          e.g. How do customers solve these problems today?

    \subsection{Problem Interview}
      Interviews will generally follow the order below:
      \begin{itemize}
        \item Welcome
        \item Collect demographics (i.e. test customer segment)
        \item Tell a story (i.e. set a problem context)
        \item Problem ranking
        \item Explore customer's worldview
        \item Wrapping up (i.e. hook and ask)
        \item Document results
      \end{itemize}

    \subsection{Processing Results}
      It is important to home in on early adopters. As more information is gathered, it may help to refine the problem or pivot the hypothesis.
      It helps to continue gathering results until they are fairly consistent. Good indications that enough testing has been done are that
      you can identify the demographics of an early adopter, have a must-have (i.e. critical problem to work on first) problem, and can describe
      how customers currently solve this problem.

\end{document}

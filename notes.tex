\documentclass[12pt,titlepage]{article}
\usepackage[margin=1in]{geometry}

\begin{document}
  \begin{titlepage}
    \vspace*{\fill}
    \centering

    \textbf{\Huge SE 463 Course Notes} \\ [0.4em]
    \textbf{\Large Software Requirements Specification and Analysis} \\ [1em]
    \textbf{\Large Michael Socha} \\ [1em]
    \textbf{\large 4A Software Engineering} \\
    \textbf{\large University of Waterloo} \\
    \textbf{\large Spring 2018} \\
    \vspace*{\fill}
  \end{titlepage}

  \newpage 

  \tableofcontents

  \newpage

  \section{Course Overview}

  \section{Hypothesis Testing}
    Purpose of hypothesis testing is to validate or disprove guesses in a project's business model. Hypothesis testing is generally
    conducted by performing pass/fail experiments through market research or customer interviews. Hypothesis being disproved can lead
    to learning, after which it can be pivoted based on new understanding.

    \subsection{Hypothesis}
      A hypothesis is a tentative guess about a phenomenon of interst that is testable and falsifiable. The phenomenon of interest
      (e.g. characteristic of customer problem) are considered to be dependent variables, while independent variables affect the
      phenomenon of interest. A good hypothesis must be testable (i.e. possible to observe effects of independent variables on
      dependent variables) and falsifiable.

      \subsection{Falsifiable Hypothesis}
        A falsifiable hypothesis is a statement that can easily be proven wrong. Specific and testable statements are more falsifiable
        than vague ones. Falsifiable hypothesis often have somewhat of a universaly quantification aspect (e.g. all, most, etc), rather
        than merely existential quantification.

    \subsection{Testing Risks}
      Uncertainty simply means that there are multiple possilities. Risk tends to involve uncertainty, with some of the possibilities
      leading to failure. The three largest risks to a project are:

        \subsubection{Customer Risk}
          e.g. Is there a viable customer segment, and what are their risks?

        \subsubection{Product Risk}
          e.g. How do customers rank the top three problems?

        \subsubection{Market Risk}
          e.g. How do customers solve these problems today?

    \subsection{Problem Interview}
      Interviews will generally follow the order below:
      \begin{itemize}
        \item Welcome
        \item Collect demographics (i.e. test customer segment)
        \item Tell a story (i.e. set a problem context)
        \item Problem ranking
        \item Explore customer's worldview
        \item Wrapping up (i.e. hook and ask)
        \item Document results
      \end{itemize}

    \subsection{Processing Results}
      It is important to home in on early adopters. As more information is gathered, it may help to refine the problem or pivot the hypothesis.
      It helps to continue gathering results until they are fairly consistent. Good indications that enough testing has been done are that
      you can identify the demographics of an early adopter, have a must-have (i.e. critical problem to work on first) problem, and can describe
      how customers currently solve this problem.

\end{document}

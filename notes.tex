\documentclass[12pt,titlepage]{article}
\usepackage[margin=1in]{geometry}
\usepackage{indentfirst}

\begin{document}
  \begin{titlepage}
    \vspace*{\fill}
    \centering

    \textbf{\Huge SE 463 Course Notes} \\ [0.4em]
    \textbf{\Large Software Requirements Specification and Analysis} \\ [1em]
    \textbf{\Large Michael Socha} \\ [1em]
    \textbf{\large 4A Software Engineering} \\
    \textbf{\large University of Waterloo} \\
    \textbf{\large Spring 2018} \\
    \vspace*{\fill}
  \end{titlepage}

  \newpage 

  \tableofcontents

  \newpage

  \section{Course Overview}
    \subsection{Logistics}
      \begin{itemize}
        \item \textbf{Professor:} Joanne Atlee
        \item \textbf{Email:} jmatlee@uwaterloo.ca
      \end{itemize}

    \subsection{Topics Covered}
      Many software project failures are not due to technical reasons, but rather due to poor project specifications. This course
      focuses on requirement-related topics including:
      \begin{itemize}
        \item Different types of requirements
        \item Learning about stakeholder needs
        \item Analyzing and refining preliminary requirements
        \item Expressing requirements in various representations
        \item Managing requirement changes
      \end{itemize}

  \section{Business Model}
    \subsection{Traditional Product Model}
      A traditional ``waterfall'' style of development relies heavily on the initial implementation meeting customer needs. This
      approach often leads to requirement-related problems, since the business and development team might not understand the
      customer needs or the customer needs might change, and a waterfall model offers little room to refine these requirements.

    \subsection{Lean Product Development Model}
      A lean product development model focuses on first determining whether a project addresses an important problem (e.g. problem
      fit), and whether the project provides a solution people want (market fit). Only after these are established is a product built,
      which is usually nimble at first and then scales up.

    \subsection{Lean Canvas}
      A lean canvas is a template to help determine the initial requirements for a project, especially one in the new product or
      consulting space. The sections of a lean canvas are described below:

      \subsubsection{Customer Segments}
        This is a list of potential customers and users. It helps to have granular segments, and focus should be placed on segments likely
        to become early adopters.

      \subsubsection{Problem Segments}
        This is a list of a few top problems that each customer segment wants solved.

      \subsubsection{Unique Value Proposition}
        This is a reason why the project is unique and valuable. The focus should also be on early adopters.

      \subsubsection{Solution}
        This is a list of solutions early adopters already apply to solve this problem. From this list, some key capabilities and features
        of the software project can be determined.

      \subsubsection{Channels}
        This is a list of methods to reach the project's target customers. This is important to consider early, since these customers
        are not just useful as end users, but also as subjects of testing and experiments during development.

      \subsubsection{Revenue Streams and Costs}
        Revenue and costs should be considered early in a project. If the project is going to operate on a subscription model, it is
        a good idea to charge something from the start (with the exception of a free trial period). ``Freemium'' business models, in
        which core services are free, tend to have low conversion rates to paying features.

      \subsubsection{Key Metrics}
        This is a list of metrics to be used for measuring success.

      \subsubsection{Unfair Advantage}
        This refers to some sort of competitive advantage that is hard to replicate. Examples include an existing reputation or customer
        base, proprietary data, or patented work.

  \section{Hypothesis Testing}
    Purpose of hypothesis testing is to validate or disprove guesses in a project's business model. Hypothesis testing is generally
    conducted by performing pass/fail experiments through market research or customer interviews. Hypothesis being disproved can lead
    to learning, after which the hypothesis can be pivoted based on new understanding.

    \subsection{Hypothesis}
      A hypothesis is a tentative guess about a phenomenon of interest that is testable and falsifiable. The phenomenon of interest
      (e.g. characteristic of customer problem) are considered to be dependent variables, while independent variables affect the
      phenomenon of interest. A good hypothesis must be testable (i.e. possible to observe effects of independent variables on
      dependent variables) and falsifiable.

      \subsection{Falsifiable Hypothesis}
        A falsifiable hypothesis is a statement that can easily be proven wrong. Specific and testable statements are more falsifiable
        than vague ones. Falsifiable hypothesis often have somewhat of a universally quantification aspect (e.g. all, most, etc), rather
        than merely existential quantification.

    \subsection{Testing Risks}
      Uncertainty simply means that there are multiple possibilities. Risk tends to involve uncertainty, with some of the possibilities
      leading to failure. The three largest risks to a project are:

        \subsubsection{Customer Risk}
          e.g. Is there a viable customer segment, and what are their risks?

        \subsubsection{Product Risk}
          e.g. How do customers rank the top three problems?

        \subsubsection{Market Risk}
          e.g. How do customers solve these problems today?

    \subsection{Problem Interview}
      Interviews will generally follow the order below:
      \begin{itemize}
        \item Welcome
        \item Collect demographics (i.e. test customer segment)
        \item Tell a story (i.e. set a problem context)
        \item Problem ranking
        \item Explore customer's worldview
        \item Wrapping up (i.e. hook and ask)
        \item Document results
      \end{itemize}

    \subsection{Processing Results}
      It is important to home in on early adopters. As more information is gathered, it may help to refine the problem or pivot the hypothesis.
      It helps to continue gathering results until they are fairly consistent. Good indications that enough testing has been done are that
      you can identify the demographics of an early adopter, have a must-have (i.e. critical problem to work on first) problem, and can describe
      how customers currently solve this problem.

  \section{Stakeholders}
    A stakeholder is considered to be anyone who has a stake in a project's ultimate success or failure. Stakeholders an include the development
    team, people in the operational work area, the containing business, and forces from the outside world.

    \subsection{Owners/Clients/Champions}
      The owner/client is the group paying for the software to be developed. They are usually considered to be the ultimate/champion stakeholder,
      since they often have the final say in a project. Examples include a client in a consultancy project, or the company developers are working
      for in an internal project.

    \subsection{Customers}
      A customer buys a project after its completion. The customer may be the same person as the owner or user.

    \subsection{Users}
      Users tends to be experts on the work the system is performing, and experts on any existing systems or competing products. Users typically
      have specific needs a product should satisfy.

      \subsubsection{User Classes}
        To meet various user needs, it helps to categorize users by their differences, including:
        \begin{itemize}
          \item Access privilege and security levels
          \item Tasks regularly performed
          \item Features used
          \item Frequency of use
          \item Application domain expertise, technical expertise
          \item Platform of use
          \item Preferred language
          \item Disfavored users (users who should not have access for security, legal or safety reasons, such as kids on some social media platforms)
        \end{itemize}

    \subsection{Personas}
      Personas are resemblances of actual users that can be constructed when real users are hard to interview (e.g. not numerous, too numerous). Personas
      should imitate the key details of important user classes. Enough details should be provided to make the persona seem realistic. Effectively built
      personas can:
      \begin{itemize}
        \item Guard against building a product just from the developer's perspective
        \item Guard against adapting to ``elastic'' users, and instead focus on keeners and early adopters
      \end{itemize}

    \subsection{Domain Expert}
      Domain experts understand the problem domain very well, and familiar with typical users, their expectations, and potential deployment environments.
      Note that domain experts need not know much about software engineering.

    \subsection{Software Engineer}
      A software engineering is an expert on the project's development technologies. These stakeholders can represent the development teams they run, and
      are responsible for overseeing the progress of the technical and some economic aspects of the project. Software engineering can also educate customers
      how available technology can affect the requested functionality.

    \subsection{Other Stakeholders}
      Other potential stakeholders include:
      \begin{itemize}
        \item \textbf{Inspectors:} Experts on government and safety regulations
        \item \textbf{Market researchers:} Can serve as proxy for customer interviews
        \item \textbf{Lawyers:} Familiar with legal requirements and licensing
        \item \textbf{Experts on adjacent systems:} Can explain how adjacent systems can affect the project
        \item \textbf{Negative stakeholders:} Stakeholders that want a project to fail
      \end{itemize}

  \section{User Requirements}

    \subsection{Use Cases}
      A use case represents some sort of end-to-end functionality in a system, capturing both its triggering event and the complete system response. Well-defined
      use cases should not overlap much with one another. Time-triggered use cases feature time as their triggering event.

      \subsection{Actors}
        An actor is an entity that interacts with the described system, which may include users or other systems. A supporting actor provides some service to the
        described system.

        \subsubsection{Actor Generalization}
          Actors often share common use cases with one another. In these cases, actor generalization can be applied, which factors out common behavior as an
          abstract actor.

      \subsection{Modifying Use Cases}
        \begin{itemize}
          \item <<include>> is used to add a sub use case that is used in multiple other use cases
          \item <<extend>> is used to add a sub use case that is used in multiple other use cases
        \end{itemize}
        Note that well-described systems tend to use these modifiers sparingly.

      \subsection{Use-Case Descriptions}
        Textual formats known as use-case descriptions are commonly used to represent use-cases. These descriptions vary in the amount of detail provided for
        each use case.

      \subsection{User Stories}
        User stories are an alternative to use-case descriptions for describing a system. Instead of being told from the system's perspective, user stories
        provide a description of something a user wants to be able to do (i.e. user requirements). The parts of a successfully managed user story are:
        \begin{itemize}
          \item \textbf{Card:} The initial description, often phrased as ``As a <role>, I want <something> so that <benefit>''
          \item \textbf{Conversation:} Discussion with product owner to determine requirement details
          \item \textbf{Confirmation:} Criteria for determining whether implementation meets requirements (conditions of satisfaction are described
          from the system's perspective)
        \end{itemize}

        \subsubsection{Benefits of User Stories}
          The benefits of user stories include:
          \begin{itemize}
            \item Easy for stakeholders to understand
            \item User-focused style encourages discussion more than ordinary written documentation
            \item Encouraging iterative development when stories are sized accordingly
          \end{itemize}

      \subsection{Changing Requirements}
        System requirements often change during the development process. Below are key topics to consider when managing requirement changes.

        \subsubsection{Requirements Baseline}
          A requirements baseline is a set of formally reviewed and agreed core requirements for a system. Major requirement changes should pass
          through this review process, and it may be helpful to limit the rate of change to these requirements.

        \subsubsection{Unique Value Proposition}
          New requirements should not weaken a project's UVP.

        \subsubsection{Project Scope}
          When project scope is not managed well, there is a tendency for ``scope creep'' to occur. Scope creep can be mitigated by keeping a
          prioritized list of requirements, and only adding a new major ones to each release.

\end{document}
